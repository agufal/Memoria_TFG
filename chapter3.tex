% +--------------------------------------------------------------------+
% | Sample Chapter 3
% +--------------------------------------------------------------------+

\cleardoublepage

% +--------------------------------------------------------------------+
% | Replace "This is Chapter 3" below with the title of your chapter.
% | LaTeX will automatically number the chapters.
% +--------------------------------------------------------------------+

\chapter{Exploración Hardware}
\label{makereference3}

\section{Objetivos de la exploración: descubrir variedad de dispositivos y especificaciones}
\label{makereference3.1}

El primer hito a nivel técnico era encontrar una placa de desarrollo que incluyera Bluetooth de bajo consumo y permitiera conectar un sensor de acelerometría, ya que eran indispensables para la transmisión de datos entre la placa y el dispositivo móvil. 

Desde el punto de vista comercial, la intención ha sido buscar el componente con mejores prestaciones en calidad y precio con el objetivo de preparar un producto final que pudiera competir con otras opciones del mercado.

Destacamos las placas con Bluetooth incorporado y un microprocesador. En las tabla~\ref{tablaSoCBLE} observamos el análisis de cada elemento que se ha considerado.

\begin{table} % TODO marcar en un color las opciones elegidas
	\begin{center}
	\begin{tiny}
	\begin{tabular}[c]{|c|c|c|c|c|c|}
        \hline
        \multicolumn{6}{|c|}{CHIPS CON BLUETOOTH} \\
        \hline
        EMPRESA & MODELO & PROCESADOR &  FLASH &  RAM & I/O \\
        \hline
        Nordic & PTR9022 & ARM Cortex-M0 & 256 KB & 16 KB & SPI, 2-WIRE, UART \\
		Nordic & nRF51822 & ARM Cortex-M0 & 256/128KB & 32KB/16KB & SPI Master/Slave, 2-wire, UART, 31 GPIO \\
		Nordic & nRF51422 & ARM Cortex-M0 & 256/128KB & 16KB & SPI Master/Slave, 2-wire, UART, 31 GPIO \\
		TI & CC2540 / CC2541 & 8051 & 128/256KB & 8KB & 2 USART, ADC, 21 GPIO, SPI \\
		TI & CC2640F128RGZT & ARM Cortex-M3 & 128KB & 20KB & I2C, I2S, SPI, UART \\
		Cypress & 4 BLE/PRoC BLE & ARM Cortex-M0 & 128/256KB & 16/32KB & 2 SCBs, configurable como I2C, SPI o UART \\
		Cypress & PSoC 4XX7-BLE & ARM Cortex-M0 & 128KB & 16KB & I2C, SPI, UART, 36 GPIO \\
    	\hline
	\end{tabular}
	\end{tiny}
    \caption{Placas que integran radio Bluetooth Low Energy. Se destacan las alternativas seleccionadas para la siguiente fase de exploración}
    \label{tablaSoCBLE}
   \end{center}
\end{table}

Dada la naturaleza de nuestro proyecto no necesitábamos un gran poder de procesamiento ni una gran capacidad en cuanto a memoria RAM y flash, ya que el código iba a ser muy sencillo. Los dos factores principales a tener en cuenta eran que incorporase la tecnología Bluetooth Low Energy y que permitiese la conexión y el envío de datos con el acelerómetro, ya fuera mediante I2C, SPI, UART...

Realizando el primer filtro pudimos observar que algunas características eran comunes entre los SoC’s elegidos:

En materia de procesadores encontramos dos opciones comunes: Cortex-M0 de ARM (Advanced RISC Machines) y 8051 de Intel. Al ver que la mayoría de los chip elegidos montaban el procesador de ARM, claramente observamos que domina el mercado de los sistemas empotrados y las empresas han optado por utilizarlo.

El modelo de negocio utilizado por ARM consiste en la venta de licencias de sus núcleos~\cite{ARMIPCore}, lo que permite a otros fabricantes diseñar su propio System On Chip (SoC) integrando tecnología ARM.

ARM Cortex-M0 nos parecía una opción más interesante, ya que este procesador cuenta con la tecnología RISC, acrónimo de Reduced Instruction Set Computing, computación de instrucciones reducidas, cuenta con una arquitectura de 32 bits frente a los 16 de Intel 8051 y soporta la arquitectura Thumb, que mejora la densidad del código para ocupar menos espacio tanto en memoria RAM como en flash.

La cantidad de memoria RAM disponible más común para este tipo de dispositivos es de 8, 16 o 32 kilobytes. La segunda opción nos pareció más que suficiente para albergar un programa sencillo como el nuestro y las variables necesarias.

En cuanto a capacidad de almacenamiento, todas las placas cuentan con memoria flash, que pueden variar entre 128 y 256 Kilobytes. De nuevo, la complejidad de nuestro programa no iba a generar un fichero compilado de gran tamaño, y no necesitábamos guardar nada más, por lo que 128 KB nos parecieron adecuados.

En cuanto a los periféricos de entrada/salida, la mayoría de las placas de desarrollo incluyen los protocolos SPI e I2C.

El protocolo SPI consiste en el envío de la señal de reloj del maestro y en cada impulso de reloj se envía un bit al esclavo y recibe un bit de éste. Los nombres de las señales son SCK para el reloj, MOSI para el Maestro Out Esclavo In, y MISO para Maestro In Esclavo Out.

El protocolo I2C, usa dos cables, uno para el reloj (SCL) y otro para el dato (SDA). El maestro y esclavo envían datos por el mismo cable, el cual es controlado por el maestro, que crea la señal de reloj. Este protocolo utiliza direccionamiento, es decir, el primer byte enviado por el maestro se forma de 7 bits para la dirección (así que permite comunicarse con hasta 127 dispositivos) y un bit de lectura/escritura, indicando si el próximo byte vendrá desde el maestro o el esclavo. Esta tecnología se ampliará en el Capítulo~\ref{makereference4.8}.

\section{Análisis de los candidatos}
\label{makereference3.3}

Una vez recopilados los modelos de que observamos en las tablas, hemos destacado 2 placas de prototipado que cumplen con los requisitos del proyecto. Este tipo de placas ofrecen más características de las que necesitamos para el proyecto, pero suponen un primer paso para poder programar y realizar pruebas antes de pasar a chips más simples y con un menor coste. 

Por un lado escogimos la placa de desarrollo de Cypress con el kit PSoC BLE y modelo de la placa con Bluetooth \textbf{CY8CKIT-042-BLE} certificado para sistemas de bajo consumo. Es un kit provisto de un chip que ofrece un procesador ARM Cortex-M0 y capacidad y conectividad suficiente como para utilizarlo de base para el proyecto.

El entorno de desarrollo para las plataformas de Cypress es un software de escritorio llamado \textit{PSoC Creator}, en el cual podemos diseñar sistemas a través de un panel gráfico. Nos ofrece multitud de librerías disponibles para la placa utilizada y es posible codificar, compilar, y depurar código.\\

Consideramos también la placa de Nordic modelo \textbf{nRF51-DK} por ser un kit de desarrollo que ofrece el mismo procesador que Cypress, conectividad tanto I2C como SPI para realizar las pruebas con el sensor. 
Este modelo ofrece compatibilidad con la plataforma de desarrollo \textit{mbed} de ARM, es una opción que nos resultó interesante a la hora escogerla. Dispone de una gran comunidad y soporte lo cual es de agradecer.\\

En la Sección~\ref{makereference3.4} hablaremos más sobre los aspectos específicos de ambos entornos de desarrollo.

Al considerar que era más interesante tener una placa con procesador, sensores y periféricos descartamos los chips individuales y optamos por una opción más completa.

\subsection{Aspectos técnicos}
\label{makereference3.3.1}

Realizado el primer filtro destacamos las que aparecen en la tabla~\ref{tablaSoCBLE}, podemos resaltar varias características importantes que tuvimos en cuenta.

El \textbf{procesador} es una característica fundamental, es el circuito integrado central más complejo encargado de interpretar y ejecutar instrucciones y de procesar datos con un mínimo consumo de energía. 
Como podemos observar en la tabla de microprocesadores con bluetooth predomina el núcleo \textbf{ARM Cortex-M0} de bajo consumo.

\textbf{ARM} Es el acrónimo de Advanced RISC Machines (ARM), una empresa conjunta entre Acorn Computers, Apple Computer (ahora Apple Inc.) y VLSI Technology.

Son microprocesadores RISC, acrónimo de Reduced Instruction Set Computing, computación de instrucciones reducidas. Esto quiere decir que se necesitan cargar las instrucciones  de los datos de memoria a un registro antes de trabajar con ellos. Permite con esta tecnología reducir el consumo de energía con el mismo rendimiento computacional.

Estos procesadores soportan la arquitectura Thumb, que se emplea en algunos procesadores ARM para aplicaciones que necesiten mejorar la densidad de código. Consiste en usar un conjunto de instrucciones de 16 bits que es una forma comprimida del set de instrucciones ARM de 32 bits.

Otro aspecto a considerar es la \textbf{memoria flash}, esta permite la programación en el sistema. Es un tipo de memoria electrónica no volátil que se puede borrar y reprogramar fácilmente.
En estos microprocesadores se utiliza también para mantener códigos de control, por ejemplo los comandos básicos para manejos de dispositivos de entrada y salida del sistema (BIOS).
En nuestra búsqueda vimos que había de dos capacidades, de 256 KB y 128KB, en nuestro caso no esperabamos utilizar demasiado espacio en memoria, ya que sólo guardaríamos el código compilado, por lo que decidimos que la opción de menor capacidad cubría las necesidades del proyecto.

La memoria RAM no era tampoco un aspecto muy decisivo a la hora de elegir una plataforma, ya que no esperabamos una gran necesidad de memoria para el código y los datos que íbamos a manejar.
En la búsqueda pudimos encontrar de dos tipos de almacenamiento tanto de 256 KB como de 128 KB.

En cuanto a los periféricos de entrada-salida, necesitábamos que la placa tuviera un protocolo de envío de datos maestro-esclavo, la mayoría de las placas de desarrollo incluyen protocolos SPI e I2C.

El protocolo \textbf{SPI} consiste en el envío de la señal de reloj del maestro y en cada impulso de reloj se envía un bit al esclavo y recibe un bit de éste. Los nombres de las señales son SCK para el reloj, MOSI para el Maestro Out Esclavo In, y MISO para Maestro In Esclavo Out.

El protocolo \textbf{I2C} es síncrono, usa dos cables, uno para el reloj (SCL) y otro para el dato (SDA). El maestro y esclavo envían datos por el mismo cable, el cual es controlado por el maestro, que crea la señal de reloj. Este protocolo utiliza direccionamiento, es decir, el primer byte enviado por el maestro se forma de 7 bits para la dirección (así que permite comunicarse con hasta 127 dispositivos) y un bit de lectura/escritura, indicando si el próximo byte vendrá desde el maestro o el esclavo.

Tras cada byte recibido se envía una confirmación con el noveno pulso de reloj. Si el maestro quiere recibir datos sólo genera pulsos de reloj. El esclavo tiene que cuidar que el próximo bit esté listo cuando la señal de reloj es dada. Dos o más señales a través del mismo cable pueden causar conflicto, y habría problemas si un dispositivo envía un 1 lógico al mismo tiempo que otro envía un 0. Por tanto el bus es "cableado" con resistencias de pull-up para poner el bus a nivel alto, y son los dispositivos los que fuerzan niveles bajos para enviar un ‘0’ lógico. Si quieren enviar un nivel alto simplemente lo comunican al bus.

\section{Plataformas escogidas}
\label{makereference3.4}

Rápidamente observamos dos grandes empresas especializadas en el sector como son Nordic y Cypress. Tienen gran variedad de microprocesadores y placas de desarrollo que cumplen con nuestras expectativas.

Aunque el precio es más elevado, nos decantamos por el kit de desarrollo de Nordic (\textbf{nRF51-DK}) este que incluye Bluetooth Smart, y protocolo de operación  ANT a 2.4GHz. Incorpora un núcleo ARM Cortex-M0 32-bit  como la mayoría de los chips que buscábamos.Una memoria flash a 256/128KB con RAM de 32KB/16KB para mejorar el rendimiento de las aplicaciones.

El kit permite el acceso a todas las interfaces de entrada y salida como SPI Master/Slave, 2-wire, UART y 31 GPIO a través de conectores. Tiene 4 LED’s y ofrece también 4 botones que son programables por el usuario. 

Utiliza un cable micro USB 2.0 para conectarse a uno de los puertos USB de el PC. Esto proporciona alimentación a la placa, y es compatible con la programación de destino.

\begin{figure}[h]%t=top, b=bottom, h=here
	\centering
    \includegraphics[scale=0.5]{figures/nRF51DK_pines.png} % TODO hacer que esto no quede horrible

    \caption[Esquema de la disposición de pines de la placa nRF51-DK de Nordic]{Esquema de la disposición de pines de la placa nRF51-DK de Nordic}

   \label{figuraNordicNRF51}
\end{figure}

La  carga de programas resulta sencilla, ya que solo debemos abrir un explorador de archivos de Windows. Confirmar que el nRF52 DK ha aparecido como una unidad extraíble llamado "JLINK". Esto permite programar el chip a bordo. Podremos cargar los programas a la placa arrastrando y soltando hacia la unidad.

\textbf{ARM Mbed} es una plataforma gratuita de prototipado rápido y experimentación con microcontroladores ARM. Provee a los desarrolladores una plataforma productiva para realizar pruebas conceptuales y prototipos en lenguaje de programación C/C++. Incluye amplia variedad de librerías, tutoriales y ejemplos, además de contar de una gran comunidad online de más de 130.000 desarrolladores de software, cuyos códigos son habitualmente accesibles a toda la comunidad.

Otro chip con procesador que nos pareció interesante fue el modelo CY8CKIT-042-BLE de la empresa Cypress que soporta 2 dispositivos: PSoC 4 BLE y  PRoC BLE.

El modelo escogido es el PSoC 4 BLE que provee de una completa solución para conectividad Bluetooth Low Energy. También monta una procesador ARM Cortex-M0 con una memoria flash de 128kB / 256kB y RAM de 16kB / 32kB. Dispone de 4 TCPWM1, 2 SCBs2, LCD4, I2S5, and 36 GPIOs.

\begin{figure}[h]%t=top, b=bottom, h=here
	\centering
    \includegraphics[width=\linewidth]{figures/cypress_psoc.PNG} % TODO hacer que esto no quede horrible

    \caption[TODO]{TODO}

   \label{figuraCypressPeque}
\end{figure}

\begin{figure}[h]%t=top, b=bottom, h=here
	\centering
    \includegraphics[scale=0.5]{figures/cypress_placa_desarrollo} % TODO hacer que esto no quede horrible

    \caption[TODO]{TODO}

   \label{figuraCypressGrande}
\end{figure}

El kit incluye una memoria extraible que conecta la placa por Bluetooth llamado Dongle USB CySmart (BLE Dongle) que se empareja con la herramienta de emulación principal CySmart. El emparejamiento con un entorno Windows hace que sea un potente entorno de depuración de Bluetooth LE.

\begin{figure}[h]%t=top, b=bottom, h=here
	\centering
    \includegraphics[scale=0.5]{figures/cypress_dongle.png} % TODO hacer que esto no quede horrible

    \caption[TODO]{TODO}

   \label{figuraCypressDongle}
\end{figure}

Este kit de desarrollo de Cypress es compatible con diseños a nivel de sistema mediante PSoC Creator, un software de desarrollo que contiene numerosos proyectos de ejemplo para proporcionar diseños integrados de señal mixta Bluetooth de baja energía, el lenguaje utilizado es C.

Es sencilla diseñar pues con arrastrar y soltar los componentes se añaden al panel principal obteniendo máxima flexibilidad de diseño.

\section{Conclusión}
\label{makereference3.5}

Tanto la placa nRF51-DK de Nordic como CY8CKIT-042-BLE de Cypress son dos placas de desarrollo perfectas para iniciar un proyecto con conectividad Bluetooth, las dos tienen idénticas características, conectores y periféricos, se les puede incluir una pila de botón CR2032 para darles autonomía y les respalda un software para desarrollo de las mismas.

En este punto tuvimos cierta controversia, ya que el software de Mbed de Nordic nos da la posibilidad de compilar código C++ en cualquier PC con internet gracias a su versión web. Nos ofrece una gran comunidad que es de agradecer y mucho contenido de ejemplos y tutoriales en la web oficial. Por contra no tiene modo depuración por pasos y eso dificulta su seguimiento y depuración. 

Por otro lado, PSoC Creator de Cypress nos ofrece un completo programa de escritorio muy visual al diseñar circuitos y un sistema “drag and drop” muy intuitivo. Permite modo depuración facilitando su programación. Por contra nos hemos encontrado con ciertas dudas que han sido difíciles de resolver en foros debido al poco contenido sobre el tema.