
\cleardoublepage

\chapter{Instalación de entornos de desarrollo y primeros desarrollos HW}
\label{makereference4}

Antes de comenzar programando la funcionalidad principal de este proyecto, tuvimos que aprender a utilizar herramientas y conceptos como los entornos de desarrollo de las placas de Nordic y de Cypress o los protocolos SPI e I2C y el propio BLE, con los que no habíamos trabajado nunca, por lo que primero realizamos una serie de pruebas para familiarizarnos y así decidir cuál de ellas elegiríamos finalmente para el proyecto.

\section{Pruebas iniciales con Cypress y Nordic}
\label{makereference4.1}

Una vez tuvimos las placas, el primer paso fue buscar códigos de ejemplo con funcionalidades parecidas a lo que íbamos a tratar en el proyecto.

\textbf{Cypress} pone a disposición de cualquier desarrollador que desee realizar pruebas un repositorio en GitHub con 100 proyectos que sirven de ejemplo para utilizar la funcionalidad BLE de sus dispositivos PSoC. Aparte han desarrollado una aplicación Android llamada \textit{CySmart} para comprobar el funcionamiento de algunos de estos ejemplos.

\textbf{mbed} dispone de un repositorio propio donde cualquier usuario puede subir proyectos para cualquier dispositivo compatible con mbed. Estos proyectos son muy sencillos de buscar e importar desde el propio compilador. ARM mbed también dispone de una app Android, ésta llamada \textit{nRF Master Control Panel}, que permite hacer algunas pruebas.

Como primera toma de contacto con las posibilidades de las placas para comunicarse físicamente con otros dispositivos realizamos un pequeño circuito consistente en una fotoresistencia conectada a un conversor analógico-digital MCP3008, capaz de convertir una entrada analógica de voltaje en un valor binario, el cual se transmitiría por el bus SPI.

Una vez creado el circuito comprobamos con un voltímetro que dejar pasar menos luz sobre la fotoresistencia disminuía la cantidad de voltaje. El rango de voltaje que pudimos comprobar fue de 0,82 V a 1,82 V con luz ambiente.

\begin{figure}[h]%t=top, b=bottom, h=here
	\centering
    \includegraphics[scale=0.5]{figures/mcp3008_esquema.PNG} % TODO hacer que esto no quede horrible
    \caption[Diagrama de pines para el conversor A/D MCP3008]{Diagrama de pines para el conversor A/D MCP3008}
   	\label{figuraMCPEsquema}
\end{figure}

\begin{figure}[h]%t=top, b=bottom, h=here
	\centering
    \includegraphics[scale=0.5]{figures/circuito_fotoresistencia.jpg} % TODO hacer que esto no quede horrible
    \caption[Circuito realizado para la prueba MCP3008]{Circuito realizado para la prueba MCP3008, con una fotoresistencia.}
   	\label{figuraMCPEsquema}
\end{figure}

Para realizar la transferencia de datos por SPI colocamos como master la placa de desarrollo y como esclavo el circuito de la fotoresistencia. Conectamos los 4 pines correspondientes en cada placa: para la trasmisión de datos (MOSI, MISO), frecuencia de reloj (SCLK) y selección del esclavo (nSS). Ambos entornos de desarrollo ofrecen ejemplos de módulo SPI con lo que fue fácil hacer funcionar el sistema una vez conectados los pines correctos.

Para comprobar en un principio si se recibían correctamente los datos, en la placa nRF51-DK de Nordic utilizamos sus 4 LED’S verdes, apagándolos o encenciéndolos según el valor ya digitalizado que recibe. De igual forma al probarlo con PSoC BLE de Cypress interactuamos con su LED RGB, utilizando los colores verde, azul y rojo para representar los diferentes datos que recibía del circuito.

Estas primeras pruebas nos sirvieron para comprobar la correcta recepción de datos, pero el objetivo era llevarlos a una aplicación desarrollada en Android. Por tanto nuestra primera toma de contacto con el desarrollo de la aplicación Android (Figura~\ref{figuraAPPPrueba}) que haría uso del módulo BLE fue en este momento, centrándonos más en la funcionalidad  que en el diseño. Esta aplicación funcional nos permitió probar conceptos como el de activar desde dentro de una app el Bluetooth del móvil, realizar un escaneo para detectar los dispositivos con Bluetooth cercanos, conectarse a éstos y enviar datos. Para verificar que funcionaba correctamente, escribimos mensajes de log  que se pueden visualizar en el entorno de desarrollo Android Studio.

\begin{figure}[h]%t=top, b=bottom, h=here
	\centering 	
    \includegraphics[scale=0.8]{figures/app_piloto2.PNG} % TODO hacer que esto no quede horrible
   	\caption[Aplicación piloto para comprobar la conexión Bluetooth]{Aplicación piloto para comprobar la conexión Bluetooth}
   	\label{figuraAPPPrueba}
\end{figure}

El siguiente paso hacia nuestro aprendizaje fue conectar por I2C el Acelerómetro XTRINSIC-SENSE-BOARD Element14. Documentándonos desde su datasheet, conectamos los pines correctos con la placa, e iniciamos las pruebas de recepción de datos.
Nuevamente, la galería de proyectos que oferece el entorno mbed nos permitió desarrollar el código de comunicación I2C sin mucha dificultad.

% Ponemos aqui que la de Cypress no nos funcionaba y decidimos dejarla?

\section{Acerca de los entornos de desarrollo}
\label{makereference4.2}

\subsection{ARM mbed}
\label{explicacionARMmbed}

La interfaz del compilador mbed es \textit{online}, permitiendo, tras registrarse, acceder al entorno desde cualquier navegador en cualquier dispositivo con una conexión a Internet sin importar el sistema operativo del ordenador en el que se programe. El lenguaje de programación utilizado es C++, y una vez elegido a qué dispositivo va orientado el código, mbed se encarga de toda la configuración a bajo nivel, por lo que, por ejemplo, sacar una señal a través de un pin se puede realizar tan fácimente como declarar una variable \textit{DigitalOut} con su correspondiente número de pin.

Una vez tenemos un código listo, mbed permite compilarlo y descargar el archivo hexadecimal a nuestro sistema si no ha habido errores de compilación. Para cargarlo en la placa de Nordic, basta con conectar ésta al ordenador por USB, lo que la abrirá en nuestro sistema como una \textit{unidad flash} en la que podremos copiar el archivo compilado. Si ha ocurrido algún problema en la carga, o el programa cargado ha fallado por un error de ejecución, se creará en esa unidad extraíble un archivo de \textit{log} con información sobre el error.

\begin{figure}[h]%t=top, b=bottom, h=here
	\centering 	
    \includegraphics[width=\textwidth]{figures/mbed_compiler.PNG} % TODO hacer que esto no quede horrible
   	\caption[Entorno de desarrollo mbed]{Entorno de desarrollo mbed}
   	\label{figuraMbedCompiler}

\end{figure}

La plataforma mbed está muy orientada a la colaboración entre desarrolladores, e implementa un repositorio propio donde cualquier usuario puede subir su código y otros usuarios pueden importarlos de forma sencilla a sus proyectos. Esto se hace evidente teniendo en cuenta que en la propia interfaz se encuentra un botón \textit{Import} que abre un buscador para encontrar códigos o librerías a través de palabras clave, lo cual resultó de gran ayuda en el período de pruebas y en la fase final, como se comentaba anteriormente.\\

Como puntos negativos se puede resaltar que mbed no cuenta con ninguna herramienta de \textit{debug}, por lo que nos vimos obligados a comprobar el correcto funcionamiento de nuestros programas a través de LEDs o de \textit{printf}s, que nos permitían enviar cadenas de texto a través del puerto USB del dispositivo. También cabe destacar que, aunque el hecho de ser online supone una ventaja a la hora de trabajar desde varios sistemas, resulta un problema si no se dispone de conexión a Internet en un determinado momento.

\subsection{Cypress PSoC Creator}
\label{explicacionPSoCCreator}

En el caso del kit PSoC 4 BLE de Cypress, la plataforma de desarrollo utilizada es propia de la marca Cypress. Esto quiere decir que está orientada exclusivamente para la programación de las placas de esta empresa.

Este entorno de desarrollo consiste en una aplicación de escritorio específica para el sistema operativo Windows, la cual ofrece además una serie de programas de configuración para actualizar el \textit{firmware} de la placa de desarrollo y comprobar el estado del enlace con el ordenador. El lenguaje de programación que se utiliza es C, y utiliza un sistema de \textit{drag and drop} para especificar las funcionalidades que se van a utilizar y definir su interconexión y su configuración. \\

\begin{figure}[h]%t=top, b=bottom, h=here
	\centering 	
    \includegraphics[width=\textwidth]{figures/psoc_compilador.PNG}
   	\caption[Entorno de desarrollo PSoC Creator]{Entorno de desarrollo PSoC Creator}
   	\label{figuraPSoCCreator}
\end{figure}

Este sistema permite, por ejemplo, buscar la componente de Bluetooth Low Energy, arrastrarla hasta la vista de diseño, y elegir la opción \textit{Generar Aplicación} para generar el código necesario para interactuar con BLE, creando un archivo de encabezado \textit{BLE.h} y uno de código \textit{BLE.c}.\\

Cypress también pone a disposición del desarrollador tutoriales en vídeo y una serie de ejemplos de código denominada \textit{100 projects in 100 days}~\cite{100Projects}, subida a un repositorio en GitHub, de las que pudimos tomar ejemplo para avanzar con las pruebas.
Sin embargo, la comunidad de desarrolladores era más pequeña que la de \emph{mbed}, probablemente por ser esta plataforma específica para productos de Cypress, mientras que en el ecosistema de \emph{mbed} se encuentran multitud de fabricantes.

A diferencia de \emph{mbed}, PSoC Creator sí ofrece herramientas de \textit{debug}, con puntos de ruptura y visualización del estado de las variables.

\section{Conclusión tras las pruebas}
\label{makereference4.3}

Tras realizar las pruebas definidas en la Sección~\ref{makereference4.1}, aprendimos a desarrollar rápidamente códigos para lograr la funcionalidad deseada. 

Los ejemplos de uso de Bluetooth Low Energy nos ayudaron a comprender los pasos que se debían dar para establecer correctamente una conexión, aunque tuvimos que informarnos más para comprender qué era exactamente lo que sucedía en cada paso.

Una vez entendimos el procedimiento para conectar por SPI el circuito mencionado anteriormente y la lógica que había que seguir para programar la interacción tanto con la placa de Cypress como la de Nordic, realizar la transición a I2C para conectar el acelerómetro fue un paso sencillo, pues las API de ambas placas gestiona de manera similar los dos protocolos.

Como ya mencionamos en el Capítulo~\ref{makereference3}, la decisión de continuar la fase final del proyecto con una placa u otra fue difícil, pues ambas tienen características similares y sus entornos de desarrollo tienen sus pros y sus contras, pero finalmente escogimos la placa de desarrollo nRF51-DK de Nordic por la facilidad de encontrar información y ayuda en los foros de la comunidad. Los ejemplos ofrecidos en el repositorio propio de \emph{mbed} fueron de gran ayuda para consolidar el aprendizaje con dicha placa y la propia herramienta nos pareció una gran apuesta para dispositivos IoT. Cabe destacar que la opción de Cypress podría haber sido viable perfectamente.