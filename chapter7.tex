\cleardoublepage

\chapter{Mediciones y calibrado}
\label{makereference7}

\section{Mediciones de consumo}

El objetivo principal de Bluetooth Low Energy es incorporarse a dispositivos de bajo consumo, y ser lo más eficiente posible energéticamente. Por ello decidimos comprobar la eficiencia de consumo de energía de nuestra placa, conectándola a un osciloscopio y recogiendo los datos del amperaje durante la ejecución del programa, en el que se recogen los datos del acelerómetro, se envían por BLE y se entra en un estado de \textit{espera}. La gráfica resultante se muestra en la Figura~\ref{}.

% Imagen Primeras mediciones

Se puede comprobar que no tiene un consumo muy elevado, con un máximo de 10 mA, que es cuando realiza la comunicación por BLE y un mínimo de 3 mA en los períodos inactivos. En los períodos de inactividad observamos que consume más de lo cabría esperar. Esto es debido a que pensábamos erróneamente que la función que utilizábamos para realizar la espera pondría al procesador en modo \textit{deep sleep}, que permitiría disminuir todo lo posible el uso de energía.

Buscando en la API de mbed, encontramos la forma de entrar en este modo, que se realiza con la funcion \textit{sleep( ms )}. Esta función permite desactivar todas las funcionalidades que no sean estrictamente necesarias para ahorrar energía, entrando en lo que se denomina \textit{Wait for Interrupt}, es decir, saldrá de este modo únicamente cuando detecte una interrupción. 

El punto negativo de esto es que no hace discriminación entre los tipos de interrupción, por lo que puede salir del modo \textit{sleep} aunque no se lo hayamos indicado. Para evitar esto utilizamos un \textit{callback} que se ejecutará cada segundo, activando una variable booleana. Esta variable controla un bucle en el que, si sale del modo \textit{sleep} y no es por nuestra interrupción, vuelve a ejecutar un \textit{sleep}.

Una vez implementada esta funcionalidad, volvimos a realizar la prueba de consumo y esta vez obtuvimos resultados más satisfactorios, como se puede apreciar en la Figura~\ref{}

% Imagen segundas mediciones

%% Cómo se calcularía el tiempo que nos puede durar la pila con los datos?

\section{Calibrado}

Debido a la presencia constante de la aceleración de la gravedad en cualquier objeto, el acelerómetro recoge siempre este valor, repartido entre los ejes (x, y, z). En llano, la aceleración o deceleración se produce sólo en el eje x, pero cuando nos hallamos en una pendiente este valor se reparte entre los ejes (x, z), por lo que parte de la aceleración se juntará con el valor de la gravedad.

Para una correcta medición, debemos eliminar el factor gravedad mediante un calibrado. Una manera de hacerlo es hallar la magnitud del vector y eliminar 1 g al resultado. El problema de esto es que ya que la magnitud no tiene signo, perdemos la noción de estar acelerando o decelerando.

\cleardoublepage

\chapter{Conclusiones y trabajo futuro}
\label{makereference8}

Este proyecto nos ha permitido explorar tecnologías hardware con las que no habíamos trabajado nunca, como Bluetooth Low Energy y protocolos como SPI o I2C, lo que nos ha dado unas competencias, en otros campos que no son específicamente los nuestros. Hemos aprendido a programar una placa desde cero, investigando la documentación que pone a nuestra disposición el fabricante y encontrando ejemplos para lograr los objetivos propuestos.

En el campo del desarrollo software, hemos trabajado por primera vez en el desarrollo de una aplicación Android familiarizándonos con el entorno de programación Android Studio.

La línea de aprendizaje que hemos recorrido (en la que hemos empezado haciendo numerosas pruebas pequeñas antes de continuar con el proyecto final) nos ha permitido empezar por conceptos sencillos y fáciles de entender que nos permitían avanzar rápidamente a casos mas complejos.

Finalmente hemos logrado producir una aplicación que se puede utilizar en entornos reales, con la que, fijando la placa BLE en una bicicleta, podemos realizar una ruta y visualizar correctamente los datos generados durante la misma.

Internet of Things es un concepto muy interesante, y abre las puertas para comunicar gran cantidad de dispositivos. Con la tecnología BLE podríamos realizar más proyectos útiles que hasta ahora no eran posibles.

\section{Trabajo futuro}

Aunque la aplicación es funcional y cumple con los requisitos del proyecto, siempre es interesante la idea de ampliar la utilidad de nuestra aplicación.

Un ejemplo claro es el de poder aprovechar aún más el hecho de tener la aplicación ejecutándose en un dispositivo móvil, ofreciendo la posibilidad de compartir las sesiones realizadas en las redes sociales (Twitter, Facebook, ...).

Otra posibilidad que sería de utilidad es la de fijar una luz de freno a la placa, que al detectar una deceleración se encienda para indicar a conductores, peatones, etc. que se está reduciendo la velocidad para evitar accidentes.

En relación con esto, otro punto importante de cara al futuro es el de utilizar los datos que se recogen de acelerometría para comprobar movimientos muy bruscos (como los que se darían en caso de accidente) para activar un sistema que avise a los servicios de emergencia en caso de que el usuario no responda en un breve tiempo.

Se podrían realizar más cálculos con los datos obtenidos para poder sacar más información de la ruta. Además, aprovechando que la placa XTRINSIC-SENSE-BOARD contiene, aparte del acelerómetro que usamos, un magnetómetro y un sensor de presión. 

