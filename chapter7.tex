\cleardoublepage

\chapter{Mediciones y calibrado}
\label{makereference7}

\section{Mediciones}

\subsection{Acelerometría}

\subsection{Consumo}

Para comprobar la eficiencia de consumo de energía de nuestra placa con el programa final cargado, la conectamos a un osciloscopio y recogimos los siguientes datos.

% Imagen Primeras mediciones

Se puede comprobar que  en los períodos en los que está "esperando" tiene un consumo el

\section{Calibrado}



\cleardoublepage

\chapter{Conclusiones y trabajo futuro}
\label{makereference8}

Este proyecto nos ha permitido explorar tecnologías hardware con las que no habíamos trabajado nunca, como Bluetooth Low Energy y protocolos como SPI o I2C, lo que nos ha dado unas competencias, en otros campos que no son específicamente los nuestros. Hemos aprendido a programar una placa desde cero, investigando la documentación que pone a nuestra disposición el fabricante y encontrando ejemplos para lograr los objetivos propuestos.

En el campo del desarrollo software, hemos trabajado por primera vez en el desarrollo de una aplicación Android familiarizándonos con el entorno de programación Android Studio.

La línea de aprendizaje que hemos recorrido (en la que hemos empezado haciendo numerosas pruebas pequeñas antes de continuar con el proyecto final) nos ha permitido empezar por conceptos sencillos y fáciles de entender que nos permitían avanzar rápidamente a casos mas complejos.

Finalmente hemos logrado producir una aplicación que se puede utilizar en entornos reales, con la que, fijando la placa BLE en una bicicleta, podemos realizar una ruta y visualizar correctamente los datos generados durante la misma.

Internet of Things es un concepto muy interesante, y abre las puertas para comunicar gran cantidad de dispositivos. Con la tecnología BLE podríamos realizar más proyectos útiles que hasta ahora no eran posibles.

\section{Trabajo futuro}

Aunque la aplicación es funcional y cumple con los requisitos del proyecto, siempre es interesante la idea de ampliar la utilidad de nuestra aplicación.

Un ejemplo claro es el de poder aprovechar aún más el hecho de tener la aplicación ejecutándose en un dispositivo móvil, ofreciendo la posibilidad de compartir las sesiones realizadas en las redes sociales (Twitter, Facebook, ...).

Otra posibilidad que sería de utilidad es la de fijar una luz de freno a la placa, que al detectar una deceleración se encienda para indicar a conductores, peatones, etc. que se está reduciendo la velocidad para evitar accidentes.

En relación con esto, otro punto importante de cara al futuro es el de utilizar los datos que se recogen de acelerometría para comprobar movimientos muy bruscos (como los que se darían en caso de accidente) para activar un sistema que avise a los servicios de emergencia en caso de que el usuario no responda en un breve tiempo.

Se podrían realizar más cálculos con los datos obtenidos para poder sacar más información de la ruta. Además, aprovechando que la placa XTRINSIC-SENSE-BOARD contiene, aparte del acelerómetro que usamos, un magnetómetro y un sensor de presión. 

